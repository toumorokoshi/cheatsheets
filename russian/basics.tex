\documentclass[10pt,landscape]{article}
\usepackage{multicol}
\usepackage{calc}
\usepackage{ifthen}
\usepackage[landscape]{geometry}
\usepackage{amsmath,amsthm,amsfonts,amssymb}
\usepackage{color,graphicx,overpic}
\usepackage{hyperref}
% need nice looking tables
\usepackage{booktabs}
% Need colors up in here
\usepackage{color, colortbl}
% Color name set, with functionality fo
\usepackage[usenames, dvipsnames, table]{xcolor}

% Activate russian compatability
\usepackage[T2A,T1]{fontenc}
\usepackage[utf8]{inputenc}
\usepackage[russian,english]{babel}

\pdfinfo{
  /Title (basics.pdf)
  /Creator (TeX)
  /Producer (pdfTeX 1.40.0)
  /Author (toumorokoshi)
  /Subject (Example)
  /Keywords (pdflatex, latex, pdftex, tex)}

% This sets page margins to .5 inch if using letter paper, and to 1cm
% if using A4 paper. (This probably isn't strictly necessary.)
% If using another size paper, use default 1cm margins.
\ifthenelse{\lengthtest { \paperwidth = 11in}}
    { \geometry{top=.5in,left=.5in,right=.5in,bottom=.5in} }
    {\ifthenelse{ \lengthtest{ \paperwidth = 297mm}}
        {\geometry{top=1cm,left=1cm,right=1cm,bottom=1cm} }
        {\geometry{top=1cm,left=1cm,right=1cm,bottom=1cm} }
    }

% define some colors here
\definecolor{railsred}{RGB}{176, 43, 44}
% defining column types for colors
\newcolumntype{g}{>{\columncolor{Gray}}c}
% defines row coloring here
\rowcolors{2}{railsred}{Magenta}

% Turn off header and footer
\pagestyle{empty}

% Redefine section commands to use less space
\makeatletter
\renewcommand{\section}{\@startsection{section}{1}{0mm}%
                                {-1ex plus -.5ex minus -.2ex}%
                                {0.5ex plus .2ex}%x
                                {\normalfont\large\bfseries}}
\renewcommand{\subsection}{\@startsection{subsection}{2}{0mm}%
                                {-1explus -.5ex minus -.2ex}%
                                {0.5ex plus .2ex}%
                                {\normalfont\normalsize\bfseries}}
\renewcommand{\subsubsection}{\@startsection{subsubsection}{3}{0mm}%
                                {-1ex plus -.5ex minus -.2ex}%
                                {1ex plus .2ex}%
                                {\normalfont\small\bfseries}}
\makeatother

% Define BibTeX command
\def\BibTeX{{\rm B\kern-.05em{\sc i\kern-.025em b}\kern-.08em
    T\kern-.1667em\lower.7ex\hbox{E}\kern-.125emX}}

% Don't print section numbers
\setcounter{secnumdepth}{0}


\setlength{\parindent}{0pt}
\setlength{\parskip}{0pt plus 0.5ex}

%My Environments
\newtheorem{example}[section]{Example}
% -----------------------------------------------------------------------

\begin{document}
\raggedright
\footnotesize
\begin{multicols}{3}


% multicol parameters
% These lengths are set only within the two main columns
%\setlength{\columnseprule}{0.25pt}
\setlength{\premulticols}{0.5pt}
\setlength{\postmulticols}{0.5pt}
\setlength{\multicolsep}{0.5pt}
\setlength{\columnsep}{0.5pt}

% Title of the cheatsheet, need to find a good place to put it.
%\begin{center}
%     \Large{\underline{Ruby}} \\
%\end{center}
\subsection{Pronunciation}
\begin{tabular}{@{}llr@{}} 
\toprule
\rowcolor{red}
Letter & Pronunciation & notes \\
\foreignlanguage{russian}{А а} & a, ah (cAr, Alligator) & \\ 
\foreignlanguage{russian}{Б б} & b (Bit Bat) & \\ 
\foreignlanguage{russian}{В в} & v (Vine, Vet, Voice) & \\ 
\foreignlanguage{russian}{Г г} & g (Go, Guard, Good) & \\
\foreignlanguage{russian}{Д д} & d (Do, Dog, Double) & \\
\foreignlanguage{russian}{Е е} & hard y (Yet, Yes) & \\
\foreignlanguage{russian}{Ё ё} & yo (York, Yolk, Yonder) & \\
\foreignlanguage{russian}{Ж ж} & zh (meaSure, pleaSure) & \\
\foreignlanguage{russian}{З з} & z (Zoo, Zodiac) & \\
\foreignlanguage{russian}{И и} & ee (mE, sEE, mEEt) & \\
\foreignlanguage{russian}{Й й} & soft y (boY, oYster, Yes) & \\
\foreignlanguage{russian}{К к} & k (Kitten, Kite) & \\
\foreignlanguage{russian}{Л л} & l (Lamp, Lock) & \\
\foreignlanguage{russian}{М м} & m (Map, My, Moon) & \\
\foreignlanguage{russian}{Н н} & n (Not, Nose, Nine) & \\
\foreignlanguage{russian}{О о} & o (mOre, AUdio) & \\
\foreignlanguage{russian}{П п} & p (Pet, Pot, Point) & \\
\foreignlanguage{russian}{Р р} & r (thRiller, rolling the r) & \\
\foreignlanguage{russian}{С с} & s (See, Sun, Soft) & \\
\foreignlanguage{russian}{Т т} & t (Tip, Task, Time) & \\
\foreignlanguage{russian}{У у} & oo (bOOk, bOOth) & \\
\foreignlanguage{russian}{Ф ф} & f (Face, Fact, Force) & \\
\foreignlanguage{russian}{Х х} & (loch with scottish accent) & \\
\foreignlanguage{russian}{Ц ц} & ts (siTS, thaT'S) & \\
\foreignlanguage{russian}{Ч ч} & ch (CHip, CHair) & \\
\foreignlanguage{russian}{Ш ш} & sh (SHip, SHut, SHe) & \\
\foreignlanguage{russian}{Щ щ} & sh + ch (freSH CHicken) & \\
\foreignlanguage{russian}{Ы ы} & kind of like 'uee' & \\
\foreignlanguage{russian}{Ы ы} & s (See, Sun, Soft) & \\
\foreignlanguage{russian}{Э э} & eh (mEt, bEt) & \\
\foreignlanguage{russian}{Ю ю} & yu (YOU, Use) & \\
\foreignlanguage{russian}{Я я} & ya/yuh (YArd, YUmmy) & \\
\foreignlanguage{russian}{Ь} & adds a little 'yuh' (softener) & \\
\foreignlanguage{russian}{Ъ} & hardens & \\
\end{tabular}
\subsection{String Method}
\begin{tabular}{@{}llr@{}} 
\toprule
capitalize(!) & Capitalizes string \\
center & \\
chomp(!) & \\
concat X& concats X to string\\
count & \\
capitalize(!) & Capitalizes string \\
capitalize(!) & Capitalizes string \\
capitalize(!) & Capitalizes string \\
Emu  & stuffed & 33.33 \\
Armadillo & frozen & 8.99 \\ \bottomrule
\end{tabular}

\section{Section 2}
Text 2

\section{Section 3}
Etc.

% You can even have references
\rule{0.3\linewidth}{0.25pt}
\scriptsize
\bibliographystyle{abstract}
\bibliography{refFile}
\end{multicols}
\end{document}
